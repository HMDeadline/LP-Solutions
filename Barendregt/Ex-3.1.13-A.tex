
	\begin{enumerate}[label = \Roman*)]
		\item 
			\begin{enumerate}
				\item 
				we insert all the variables into the context at once, hence by putting
				\linebreak
				$
					\Gamma := x : \sigma \rightarrow \tau \rightarrow \rho, \; y : \sigma \rightarrow \tau, \; z : \sigma
				$ then we have : 
				\newline\newline
					\begin{prooftree}
						\hypo{\Gamma \vdash x : \sigma \rightarrow \tau \rightarrow \rho}
						\hypo{\Gamma \vdash z : \sigma}
						\infer2{\Gamma \vdash xz : \tau \rightarrow \rho}
						\hypo{\Gamma \vdash y : \sigma \rightarrow \tau}
						\hypo{\Gamma \vdash z : \sigma}
						\infer2{\Gamma \vdash yz : \tau}
						\infer2{\Gamma \vdash xz(yz) : \rho}
						\infer1{\vdash \lambda xyz.xz(yz) : (\sigma \rightarrow \tau \rightarrow \rho)\rightarrow (\sigma \rightarrow \tau) \rightarrow (\sigma \rightarrow \rho)}
					\end{prooftree}
					\item 
					We know that K : $\sigma \rightarrow \tau \rightarrow \sigma$ and also in the part (a) we proved that for all $\sigma, \tau, \rho, \in \mathbb{T}$ we have S : $(\sigma \rightarrow \tau \rightarrow \rho)\rightarrow (\sigma \rightarrow \tau) \rightarrow (\sigma \rightarrow \rho)$, hence we can say S : $(\sigma \rightarrow \tau \rightarrow \sigma)\rightarrow (\sigma \rightarrow \tau) \rightarrow (\sigma \rightarrow \sigma)$ or simply just put $\rho = \sigma$. So we have
					\\\\
					\begin{prooftree}
						\hypo{\vdash S : (\sigma \rightarrow \tau \rightarrow \sigma)\rightarrow (\sigma \rightarrow \tau) \rightarrow (\sigma \rightarrow \sigma)}
						\hypo{\vdash K : \sigma \rightarrow \tau \rightarrow \sigma}
						\infer2{\vdash SK : (\sigma \rightarrow \tau) \rightarrow \sigma \rightarrow \sigma}
					\end{prooftree}
					\item 
					One can easily show that K : $(\sigma \rightarrow \sigma)\rightarrow \tau \rightarrow (\sigma \rightarrow \sigma)$
					since we have K : $\sigma \rightarrow \tau \rightarrow \sigma$ for every $\sigma$ and $\tau$. And I : $\sigma \rightarrow \sigma$, so we have : 
					\\\\
					\begin{prooftree}
						\hypo{\vdash K : (\sigma \rightarrow \sigma)\rightarrow \tau \rightarrow (\sigma \rightarrow \sigma)}
						\hypo{\vdash I : \sigma \rightarrow \sigma}
						\infer2{\vdash KI : \tau \rightarrow \sigma \rightarrow \sigma}
					\end{prooftree}
			\end{enumerate}
			\item 
			From proposition 3.1.9 we know that every subterm of a typeable term is typeable since yz is a subterm of SK, it should be typeable. Suppose z : $\sigma$ for some $\sigma \in \mathbb{T}$
			since y is applied on z then y should be of type $\sigma \rightarrow \tau$ for some type $\tau$.
				From proposition 3.1.11 we also know that for all terms M and N if we have  M $\twoheadrightarrow_{\beta}$ N and $\vdash M : \sigma$ then we have 
				$\vdash N : \sigma$ as well. Suppose we have SK : $\tau \rightarrow \sigma \rightarrow \sigma$.
				\\
				SK = $(\lambda xyz.xz(yz))(\lambda ab.a) = \lambda yz.(\lambda ab.a)z(yz) = \lambda yz.z$ . So $\lambda yz.z : \tau \rightarrow \sigma \rightarrow \sigma$ so we must have y : $\tau$ but this is a contradiction. So $\not \vdash$ SK : $\tau \rightarrow \sigma \rightarrow \sigma$.
			\item 
			Suppose $\lambda x.xx : \sigma$ for some type $\sigma$ and x is of type $\tau$ for some $\tau$. Then by inversion of typing rules we must have 
			$$ \lambda x.xx : \sigma \Rightarrow x : \tau \vdash xx : \sigma \Rightarrow x : \tau \vdash x : \tau \rightarrow \sigma$$
			which is infeasible.
			\\
			Like the part (b) we use the fact that every subterm of a typeable term is typeable, so if KI$\omega$ is typeable then also is $\omega$. But we proved that $\omega$ isn't.  
			\end{enumerate}
